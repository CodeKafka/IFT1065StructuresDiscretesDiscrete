\documentclass[8pt]{report}
\input{preamble.tex}




\title{\Huge{Structures Discrètes}\\{IFT1065}\\{\textbf{Préparation à l'examen Intra}}}

\author{\huge{Franz Girardin}}
\date{\today}
\lstset{inputencoding=utf8/latin1}
%%%%%%%%%%%%%%%%%  Sect.                          %%%%%%%%%%%%%%%%%%%%%%%%%%%%%%%%%%%%%%%%%%%%%%%%%%%%%%%%%%%
\usepackage{helvet}
\titleformat{\chapter}
  {\fontfamily{phv}\bfseries\huge} % format
  {}                % label
  {0pt}             % sep
  {\color{myb}\huge}           % before-code



\titleformat{\section}
  {\normalfont\scshape}{\thesection}{1em}{}


% Customizing the spacing for the chapter titles
\titlespacing*{\chapter}{0pt}{-5pt}{20pt}

%\def\thesection{\alph{section}}
\usepackage{mathpazo}
\usepackage{nicematrix}
\usepackage{lscape}
\usepackage{rotating}
\begin{document}
\maketitle
\pagebreak
\tableofcontents
\pagebreak






%====================================================================

%====================================================================


\begin{multicols*}{2}
\chapter{Introduction à théorie des ensembles}
\section{Introduction aux ensembles}

Les ensembles constituent le langage dans lequel les mathématiques modernes sont exprimées. La théorie des 
ensembles formalise les opérations naturelles que l'on peut appliquer à des collections d'objets. 
\begin{Definition}{Ensemble}{}
    Un ensemble est un collection d'objets
\end{Definition}
\begin{Remarque}{}{}
  L'ordre dans lequel apparaît les objets n'est pas important. Les ensembles ne sont définis que par 
  les éléments distincts qu'ils contiennent. 
\end{Remarque}

\begin{EExample}{Égalité d'ensemble}{}
  Soit les ensembles $A = \{ 1, 2, 3 \}$ et $B = \{ 3, 1, 2 \}$, on peut donc affirmer que $A = B$ 
\end{EExample}
\begin{Syntaxe*}{Opérateur |}{}
  Les éléments d'un ensemble peuvent être donnés par des propriétés qu'il doivent satisfaire. Par exemple,
  on peut définir l'ensemble $P$ grâce à l'opérateur $|$ : 
  \begin{align*}
      P = \{ x \; | \; x \text{ est un nombre premier}  \}
  \end{align*}
\end{Syntaxe*}
\begin{Remarque}{Ensemble contenu}{}
    Un ensemble peut contenir des ensembles : 
    \begin{align}
      B = \{ \{1, \pi, 3 \} \{4, 6, \pi\}\}
    \end{align}

\end{Remarque}
%%%%%%%%%%%%%%%%%%%%%%%%%%%%%%%%%%%%%%%%%%%%%%%%%%%%%%%%%%%%%%%%%%%%%%%%%%%%%%%%%%%%%%%%%%%%%%%%%%%%%%%%%%%%%%%%%%%%%%%%
%%%%%%%%%%%%%%%%%%%%%%%%%%%%%%%%%%                Section : Ensembles connus                   %%%%%%%%%%%%%%%%%%%%%%%%%
%%%%%%%%%%%%%%%%%%%%%%%%%%%%%%%%%%%%%%%%%%%%%%%%%%%%%%%%%%%%%%%%%%%%%%%%%%%%%%%%%%%%%%%%%%%%%%%%%%%%%%%%%%%%%%%%%%%%%%%%


\section{Ensembles connus}
\begin{Syntaxe}{Les ensembles particuliers}{}
    L'ensemble des nombres premiers est définit par : 
    \begin{align}
      \mathbb{P} = \{2, 3, 5, 7, 11, 13, 17, 19, 23...\}
    \end{align}
    L'ensemble des entiers naturels non nuls est définit par :
    \begin{align}
      \mathbb{N}^* = \{ 1, 2, 3, 4, 5, ... \}
    \end{align}
    L'ensemble des entiers naturels positifs est définit par :
    \begin{align}
      \mathbb{N} = \{ \textbf{0}, 1, 2, 3, 4, \dots \}
    \end{align}
    L'ensemble des entiers non nuls est définit par :
    \begin{align}
      \mathbb{Z}^* = \{..., -3, -2, -1, 1, 2, 3, ...\} 
    \end{align}
    L'ensemble des entiers positifs et négatifs est définit par : 
    \begin{align}
      \mathbb{Z} = \{..., -3, -2, -1, \textbf{0}, 1, 2, 3, ... \}
    \end{align}
    L'ensemble des nombres rationnels est définit par $\mathbb{Q}$.
    L'ensemble des nombres irrationnels est difnit par $\mathbb{Q}^{'}$.
    L'ensemble des nombres réels est définit par $\mathbb{R}$.
    L'ensemble des nombre complexes est définit par  $\mathbb{C}$. 
\end{Syntaxe}

\begin{Definition}{Nombre rationnel}{}
  Est rationnel tout nombre pouvant être représenté comme le quotient de deux
  nombres entiers où le \textbf{dénominateur est non nul}. Le développement décimal 
  peut être fini ou infini et périodique, mais pas infini est non périodique. 
  \begin{align}
    \forall k : k = \frac{p}{q}, \; \; p \in \mathbb{Z}, \; \; q \in \mathbb{Z}  \implies k \in \mathbb{Q}
  \end{align}
\end{Definition}
\begin{EExample}{Nombres rationnels}{}
  Les nombres $0.75$, $-2.5$, et $-0.125$ peuvent sont rationnel. Ils peuvet être représenté,
  respectivement, par : 
  \begin{align*}
    \frac{3}{4}\;, \; \; \frac{-5}{2}\;, \;\; \frac{-1}{8}
  \end{align*}
\end{EExample}


\begin{Definition}{Nombre irrationnel}{}
    Est irrationnel tous nombre pouvant être exprimé sous forme de quotient de deux 
    nombres, mais dont le développement décimal est à la fois 
    infini et non périodique 
\end{Definition}

\begin{EExample}{Nombre irrationnel}{}
  $\pi$ est un nombre irrationnel. On peut approximer pi par une somme :
  \begin{align*}
  \pi \approx \lim_{{n \to \infty}} \sum_{{i=1}}^{n} \frac{1}{n} \cdot \frac{4}{1 + \left(\frac{i}{n}\right)^2}
  \end{align*}
\end{EExample}
\begin{Definition}{Nombre réel}{}
    Est réel tous nombre pouvant être représenté par un partie entière et une liste 
    finie ou infinie de décimales. L'ensemble des réels est définit par :
    \begin{align}
      \mathbb{R} = \mathbb{Q} \cup \mathbb{Q}^{'}
    \end{align}
\end{Definition}
\section{Introduction aux listes}

\begin{Definition}{Liste}{}
   Collection d'élément qui a un ordre et pouvant accepter des répétitions
\end{Definition}

\begin{Syntaxe}{Notation de liste}{}
  On peut définir une liste comme suit :
  \begin{align*}
      \left(1,2,3 \right)
  \end{align*}
\end{Syntaxe}
\begin{Remarque}{}{}
    Deux listes sont égales si elles ont les même éléments dans le même ordre :
    \begin{align*}
    \left( 1, 2, 3\right) \neq \left( 3, 2, 1\right)
    \end{align*}
    Et si deux listes ne contiennent pas le même nombres d'éléments, alors elles ne sont pas 
    les mêmes listes 
\end{Remarque}
\begin{EExample}{Différencier objets de type liste et ensemble}{}
  1. $\left( \right)$, 2. $\left(\emptyset \right)$, 3. $\left( \{ \}\right)$
  1. et 2. sont identiques 
\end{EExample}

\begin{Definition}{Produit cartésien}{}
   L'ensemble des paires qui contient les éléments du premier ensemble et du $2^e$ ensemble est le 
   produit cartésion de $A$ avec $B$ (noté $A \times B$)
   \begin{align}
     A \times B = \{ (a,b) \; | \;  a \in A \; \text{et} \; b \in B \}
   \end{align}
   
\end{Definition}

\begin{figure}[H]
    \begin{center}
        \includegraphics[width=0.45\textwidth]{ProdCart.png}
    \end{center}
\end{figure}



\section{Produit cartésien, ensemble puissance et appartenance}
\begin{EExample}{Produit cartésien}{}
  Soit $A = \{ 1, 2\}$ et $B = \{ x, y\}$. Le produit cartésien est :
  \begin{align*}
    A \times B = \{ (1,x), (1,y), (2,x), (2,y)      \}\\
    B \times A = \{ (x,1), (x,2), (y,1), (y,2)      \} 
  \end{align*}
\end{EExample}


\begin{note}{}{}
  \begin{align*}
    A \times B \neq B \times A \\ 
    A \times B \times C \neq \left( A \times B \right) \times C
  \end{align*}
  Il faut respecter l'ordre de priorité des opérations du produit cartésien 
\end{note}

\begin{qstion}{Parenthèse}{}
    Est-ce que $A \times B \times C$ = $\left(A \times B\right)$
    \textbf{Réponse} : Non. %Todo Montrer 
\end{qstion}

\begin{Definition}{Sous-ensemble}{}
  $A$ est un sous-ensemble de $B$ si tous les éléments de $A$ sont des éléments de $B$ : 
  \begin{align*}
              A \subseteq B
  \end{align*}
\end{Definition}
\begin{note}{}{}
    Un ensemble à $3$ éléments a $2^3$ sous-ensembles possibles. Donc, soit $n$, le nombre d'éléments 
    d'une ensemble, il existe $2^n$ \textbf{sous-ensembles possibles}. Par ailleurs, 
    l'ensemble vide est un sous-ensemble de n'importe quel ensemble, y compris lui-même. 

\end{note}



\begin{Definition}{Ensemble puissance}{}
  L'ensemble puissance de $A$, $\mathcal{P}\left(A\right)$ est l'ensemble de tous les sous-ensembles possibles de $A$
\end{Definition}




\begin{Remarque}{Égalité de deux ensembles}{}
    Deux ensembles sont égaux $A = B$ est équivalent à dire que $A \subseteq B$ et  
    $B \subseteq A$ 
  \end{Remarque}

  \begin{figure}[H]
    \begin{center}
        \includegraphics[width=0.45\textwidth]{EnsemblePuissance.png}
    \end{center}
    \caption{Arbre de décision pour construire l'ensemble puissance}
\end{figure}

Il est possible de trouver l'ensemble puissance en commançant par l'ensemble vide. Puisque tous les ensembles 
contiennent au moins l'ensemble vide, construire l'ensemble puissance revient à déterminer si, pour 
chaque entré de $A$, nous inclusion l'élément ou pas. 

\chapter{Structures  discrètes élémentaires}

\section{Opérations sur les ensembles}
\begin{Definitionx*}{Union}{}
  Soit deux ensembles $A = \{1, 2, 3,\}$ et $B = \{3, 4, 5\}$ on peut effectuer l'opération qui
  consiste à prendre tous les éléments soit dans l'un soit dans l'autres :
  \begin{align*}
    A \cup B = \{1, 2, 3, 4, 5\}
  \end{align*}
  De façon plus \textbf{générale}, 
  \begin{align*}
      A_1\cup A_2\cup A_3\cup\cdots\cup A_n = \\
      \{x \; |  x \in \textbf{\textit{au moins  un }}  A_i \; \forall \; 1 \leq i \leq n\}         
  \end{align*}
\end{Definitionx*} 
\begin{Definitionx*}{Intersection}{}
  L'intersection consiste à prendre uniquement les éléments qui se trouvent \textbf{à la fois} dans les 
  les deux ensembles :
  \begin{align*}
    A \cap B = \{3\} \text{ \textbf{Puisque} } A \cup B = \{x \; | \; x \in A \; ou \;x \in B \}
  \end{align*}
    De façon plus \textbf{générale}, 
  \begin{align*}
    A_1\cap A_2\cap A_3\cap\cdots\cap A_n = \\ 
    \{x \; | \; x \in \textit{\textbf{tous les ensembles} } A_i \; \forall \; 1 \leq i \leq n\}         
  \end{align*}



\end{Definitionx*}

\begin{Definitionx*}{Différence}{}
  La différence consiste à prendre tous les éléments dans le premier ensemble qui ne sont pas dans 
  le second :
  \begin{align*}
    A - B = \{1, 2\} \text{ \textbf{Puisque} } A \cup B = \{x \; | \; x \in A \; et \;x \notin B\}
  \end{align*}
\end{Definitionx*} 

\begin{EExample}{}{}
  $\mathbb{R} - \mathbb{N}$ est l'ensemble des nombres dont la partie décimale n'est pas nulle 
  ou qui sont négatrifs 
\end{EExample}


\begin{Definitionx*}{Le complément}{}
    Le complément de $X$ par rapport à $U$ est définit si : 
    \begin{align*}
      X \subseteq U \text{ comme } U - X        
    \end{align*}
\end{Definitionx*}

\begin{Concept}{Univers ou ensemble universel}{}
    Lorsqu'on discute d'un ensemble, on en parle presque toujours en référence à un autre ensemble. Par exemple, 
    lorsqu'on parle de l'ensemble des nombres premiers $P$ :
    \begin{align*}
      P = \{ 2, 3, 5, 6, 11, 13, ... \} 
    \end{align*}
    on en parle presque toujours en référence à un ensemble qui le contient, par exemple, $P \subseteq \mathbb{N}$.
    Ainsi, on dit que le plus grand ensemble est un \textbf{ensemble universel}. Il s'agit de l'ensemble auquel 
    on se réfère pour décrire certaines caractéristiques de $P$. On dit alors que $\mathbb{N}$ est 
    \textbf{l'univers} de $P$.
\end{Concept}

\begin{Definitionx*}{}{}
  Soit $A$, un ensemble ayant un comme \textbf{univers} $U$, le \textbf{complément} de $A$ dénoté $\overline{A}$ 
  est l'ensemble :
  \begin{align*}
    \overline{A} = U - A 
  \end{align*}
\end{Definitionx*}

\section{Union et intersection itéré}
\begin{Syntaxe}{Union de plusieurs ensembles}{}
  Soit les ensembles $A_1, A_2, A_3,..., A_n$, on définit \textbf{l'union de ces ensembles} comme suit : 
    \begin{align*}
      \bigcup_{i=1}^n A_i = A_1\cup A_2 \cup A_3 \cup \cdots \cup A_n 
    \end{align*}
  Par ailleurs, on définit \textbf{l'intersection de ces ensembles} comme suit :
  \begin{align*}
      \bigcap_{i=1}^n A_i = A_1\cap A_2 \cap A_3 \cap \cdots \cap A_n 
  \end{align*}
\end{Syntaxe}

\chapter{Combinatoire}
\section{Principe de multiplication (Hammack \S 4.2)}
\begin{Concept}{Description du principe de multiplication}{}
    Lorsqu'on veut construire un liste de longueur $n$ et qu'il y a $a_1$ choix possibles pour la première entrée, 
    $a_2$ choix possibles pour la secondes entrée, $a_3$ choix possible pour la troisièmes entrée et ainsi de suite. 
    Alors, \textbf{le nombre total de listes différentes} qui peuvent être construites est le produit :
    $a_1 \cdot a_2 \cdot a_3 \cdot \cdots a_n$. \\\\ 
    Autrement dit, le \textbf{nombre de listes} faisables à travers un processus spécifique est le produit 
    du \textbf{nombre de choix} pour chaque entrée.   
\end{Concept}
\begin{EExample}{Application du principe de multiplication (Hammack 4.1)}{}
    Une plaque d'immatriculation standard comprend trois lettres suivies de quatre nombres, par exemple 
    $JRB-4412$. Combien de plaque d'immatriculation standard différentes existe-t-il ? \\\\
    Il y a $26$ choix possibles—correspondant aux $26$ lettres de l'alphabet—pour le premier caractère, 
    $26$ choix pour le second et $26$ choix pour le troisième. Par ailleurs, il y a 10 nombres possibles pour 
    le $4^e$, $5^e$, $6^e$ et $7^e$ caractère. Soit $A$, l'ensemble des plaques d'immatriculation, nous avons :
    \[ |A| = 26 \cdot 26 \cdot 26 \cdot 10 \cdot 10 \cdot 10 \cdot 10 \]
\end{EExample}

\begin{EExample}{Application du principe de multiplication (Hammack 4.2)}{}
    En commandant un café, vous avez le choix entre du lait écrémé, du lait $2\%$ et du lait d'avoine; 
    un gobelet moyen ou large; et soit une ou deux doses d'espresso. Combien de boisson possibles 
    pouvez-vous commander selon les choix disponibles ? \\\\ 
    On a $3$ choix de lait, $2$ choix de gobelet et $2$ choix de dose d'espresso. Soit $B$, l'ensemble des 
    choix de boisson possibles, on a :
    \[ |B| = 3 \cdot 2 \cdot 2 = 18 \]
\end{EExample}

\section{Dénombrement de permutations}

\begin{table}[h]
  \caption {Permutations possibles selon le $n \in \mathbb{N}$ }

  \begin{center}
    \renewcommand{\arraystretch}{1.5}
    \fontfamily{flr}\selectfont
    \footnotesize
    \begin{tabular}{l|l|p{10cm}|l}
    \arrayrulecolor{blue}\hline
    \rowcolor{lightBlue}
    \textcolor{myb}{$n$} & \textcolor{myb}{Symboles} & \textcolor{myb}{Permutations de liste de longueur n} & \textcolor{myb}{$n!$}
    \\
    \hline
    \hline
    \arrayrulecolor{black}
    0 & $\{ \}$ & $()$  & 1  

    \\
    \hline
    1 & $\{a\}$ & a & 1 
    \\
    \hline

    2 & $\{a, b\}$ & $ab,\;\; ba$ & 2 
    \\
    \hline
    3 & $\{a, b, c\}$  & $abc,\;\; acb,\;\; bac,\;\; bca \;\; cab, \;\; cba$ & 6 
    \\
    \hline 
    4 & $\{ a, b, c, d\}$ &  \textit{abcd, \;\; acbd, \;\; bacd, \;\; bcad, \;\; cabd, \;\;cabd,    
              addc, \;\; acdb, \;\; badc, \;\; bcda, \;\; cadb, \;\; cbda,   
              bacd, \;\; badc, \;\; bdac, \;\; dbac, \;\; adbc, \;\; adcb,   
    dabc, \;\; dacb, \;\; dbac, \;\; dbca, \;\; dcab, \;\; dcba}  & 0 
    \\ 
    \hline 
    \vdots & \vdots & \vdots & \vdots 
    \\
    \hline 
\end{tabular}
\end{center}
\end{table}

\begin{Definitionx*}{}{}
  Si $n \in \mathbb{Z}^*$, alors $n!$ est le nombre de listes de longueur $n$ qui peuvent etre obtenues à partir de $n$ 
  symboles, sans répétition.  Ainsi, $0! = 1$ et $1! = 1$. Si $n > 1$, alors 
  $n! = n\left(n-1\right) \left(n-1\right)\cdots 3 \cdot 2 \cdot 1$.
\end{Definitionx*}

\section{Principe d'addition (Hammack \S 4.3)}
\begin{Definitionx*}{Principe d'addition}{}
  Supposons un ensemble fini $X$ qui peut être décomposé en union
  $X = X_1 \cup X_2 \cup \cdots \cup X_n$, où $X_i \cap X_j = \emptyset$ \textbf{lorsque}
  $i \neq j$. \textbf{Alors}  , 
  \begin{align*}
              |X| = |X_1| + |X_2| +\cdots+ |X_n|
  \end{align*}
\end{Definitionx*}

\begin{note}{}{}
    Le \textcolor{myb}{principe d'addition} affirme que si un ensemble peut être 
    décomposé en morceaux, la taille de cet ensemble est \textbf{la somme des tailles 
    de ses morceaux}.    
\end{note}

\section{Principe de soustraction (Hammack \S 4.3)}
\begin{Definitionx*}{Principe de soustraction}{}
    Si $X$ est un sous-ensemble d'un ensemble fini $U$, \textbf{alors} :
    \begin{align*}
      |\overline{X}| = |U| - |X|. 
    \end{align*}
    En d'autres mots, si $X \subseteq U$, \textbf{alors}   $|U - X| = |U| - |X|$
\end{Definitionx*}

\section{Factoriels et permutations}
\begin{Definitionx*}{Factoriel}{}
    Si $n$ est un \textbf{entier non négatif}, alors $n!$ is le nombre de \textcolor{red}{\textit{listes}}
    non répétitives de longueur $n$ qui peuvent être obtenues à partir de $n$ symbols. Et nous avons :
    \[ n! = n \cdot (n-1)! \]
\end{Definitionx*}


\begin{Definitionx*}{{Permutation}}{}
    Une \textbf{permutation} est un arrangement (listé) sans répétition de tous les éléments d'un ensemble. Autrement 
    dit, tous les arrangement sans répétition possibles d'une liste sont des permutations de cette liste. Une liste 
    de longueur $n$ a donc $n!$ différentes permutations.
\end{Definitionx*}

Le corollaire de cette définition est le fait que $0! = 1$ puisque qu'on peut faire uniquement 1 liste de longueur 
$n=0$ en choisissant $n=0$ symbols, soit la liste vide, ().


\begin{Definitionx*}{{k-Permutation}}{}
    Une \textbf{k-permutation} de $X$ est le nombre de listes non répétitives qu'on peut obtenir en choisissant 
    $k$ éléments parmi la liste $X$. Et l'expression $P(n, k)$ dénote le nombre de k-permutations
    d'une liste de $n$ éléments.
                                    \[ P(n, k)= n(n-1)(n-2)\cdots(n-k+1) \]
    Si $0 \leq k \leq n$, alors $P(n, k) = \dfrac{n!}{(n-k)!}$
    
\end{Definitionx*}

\section{Dénombrement de sous-ensembles (\S Hammack 4.5)}

\begin{note}{}{}
    Le \textcolor{myb}{principe de multiplication} s'applique lorsqu'on veut déterminer quelle quantité 
    de \textbf{listes} de taille $n$ on peut construire en sélectionnant $k$ entrées d'un 
    ensemble comprenant $n$ éléments. \\\\ 

    La méthode de  \textcolor{myb}{dénombrement de sous-ensembles} permet de déterminer 
    le \textbf{nombre de sous-ensembles} il est possible d'obtenir en sélectionnant $k$ 
    éléments d'un ensembles de taille $n$
\end{note}

\begin{EExample}{Principe de multiplication vs Dénombrement de sous-ensemble}{}
  Soit un ensemble $A = \{a, b, c, d, e\}$, le nombre de listes différentes qu'on peut 
  obtenir en sélectionnant $k = 2$ éléments de $A$ est donné par : 
  \begin{center}
      $N_{total} = 5 \times 4 = 20$
  \end{center}
  Et les listes sont : 
  \begin{center}
  $(a, b), (a, c), (a, d), (a, e),$ \\ 
  $(b, c), (b, d), (b, e), (c, d),$ \\ 
  $(c, e), (d, e), (b, a), (c, a),$\\ 
  $(d, a), (e, a), (c, b), (d, b),$\\ 
  $(e, b), (d, c), (e, c), (e, d)$
  \end{center}
  Par contre, il existe \textbf{uniquement $10$}   sous-ensembles de $A$, soit :
  \begin{center}
    $\{a, b\}, \{a, c\}, \{a, d\}, \{a, e\}, \{b, c\}, \{b, d\}, \{b, e\}, \{c, d\}, \{c, e\}, \{d, e\}$
  \end{center}
\end{EExample}

\begin{Definition}{Définition de $C(n, k)$}{}
  Si $n, k \in \mathbb{N}$, alors ${n \choose k}$ dénote \textbf{le nombre de sous-ensembles}   qui peuvent être obtenus en choisissant $k$ éléments d'un ensemble de taille $n$. On lit ${n \choose k}$ « $n$ choisit $k$ ». 
\end{Definition}

\begin{Theorem}{}
  Si $0 \leq k \leq n$, \textbf{alors}, ${n \choose k} =  \dfrac{n!}{k!(n-k)!}$. 
  \textbf{Autrement}, ${n \choose k} = 0$. 
\end{Theorem}

\section{Triangle de Pascal et théorème binomial (\S Hammack 4.6)}

\begin{Concept}{Formule du triangle de Pascal}{}
  $\forall n, k \in \mathbb{Z}^* \; | \; 1 \leq k \leq n$, on a la relation suivante 
  entre les nombres : 
  \begin{align*}
    { n + 1\choose k} = {n \choose k -1} + {n \choose k}       
  \end{align*}
\end{Concept}

\begin{Theorem}{}{}
  $(x+y)^n = {n \choose 0}x^n + {n \choose 1}x^{n-1}y+ {n \choose 2}x^{n-2}y^2 + 
  {n \choose 3}x^{n-3}y^3 +\cdots+ {n \choose n-1}xy^{n-1} + 
  {n \choose n}y^n$
  \begin{center}
    $$(x+y)^n = \sum_{i=0}^{n}{n \choose i}x^{n-i}y^i$$
  \end{center}
\end{Theorem}

\section{Inclusion-exclusion (\S Hammack 4.7)}
\begin{Definitionx*}{Principe d'inclusion-exclusion}{}
    Si $A$ et $B$ sont des ensembles finis, \textbf{alors} $|A \cup B| =  |A| + |B| - |A \cap B|$ 
\end{Definitionx*}


\begin{figure}[H]
    \begin{center}
        \includegraphics[width=0.30\textwidth]{InclusionExclusio.png}
    \end{center}
\end{figure}

\begin{Definitionx*}{Inclusion-Exclusion sur trois ensembles}{}
    Nous pouvons appliquer le principe sur une multitude d'ensembles : 
    \begin{align*}
             |A \cup B \cup C| =  \\ 
                            & |A| + |B| + |C| - |A \cap B| - |A \cap C| - \\ 
                            & |B \cap C| - |A \cap B \cap C|
    \end{align*}
\end{Definitionx*}
    \begin{figure}[H]
    \begin{center}
        \includegraphics[width=0.25\textwidth]{InclEx.png}
    \end{center}
\end{figure}

\section{Dénombrement avec répétition (\S Hammack 4.8)}

\begin{EExample}{}{}
    Supposons qu'on ait un sac de $5$ boules rouges $R$, et $2$ boules noirs ($N$),
    comment pouvons-nous modéliser cette situtation, sachant que 
    \begin{enumerate}
        \item Il n'y a pas d'ordre sur les boules
        \item Il n'y a pas de répétition—les boules d'une même couleur sont indistinguables. 
    \end{enumerate}
    Pour représenter la situtation on écrit alors : 
    \begin{center}
    $ \left[ R, R, R, R, R, N, N \right] $
    \end{center}
\end{EExample}

\begin{Definitionx*}{Multi-ensemble}{}
   Un multi-ensemble est une collection d'objets sans ordre particulier qui peuvent être 
   répétés. Le cardinal d'un multi-ensemble est le nombre d'objets des multi-ensemble, incluant les 
   répétitions. 
\end{Definitionx*}

\begin{Definitionx*}{}{}
    La multiplicité de $x \in A$ est le nombre de répétitions $x$ dans le multi-ensemble $A$ 
    \[ |A|_x \]
\end{Definitionx*}
\begin{note}{}{}
    Un ensemble est un cas particulier d'un multi-ensemble; il s'agit d'un multi-ensemble 
    où la multtiplicité de chaque élément est de $1$ : 
    \[ \{1, 2, 3\} = \left[1, 2, 3\right]\] 
    Cependant, 
    \[  \{1, 2, 3\} \neq \left[1, 2, 2, 3 \right] \text{ et } \emptyset = \left[ \;\right] \] 
\end{note}

\begin{EExample}{}{}
    Soit un sac avec des boules de trois couleurs différents (r, n, b), le multi-ensemble 
    représentant cette situation est donné par : 
    \[\left[ R, R, R, N, N, N, B, B, B \right]\]
    \textcolor{red}{\textbf{Question}} Combien y a-t-il de résultats possibles ?
    \begin{align*}
        \left[R, R, R \right] \; et \; \left[R, R, N \right] \\
        \left[R, N, N \right] \; et \; \left[R, N, B \right] \\
        \left[R, B, B \right] \; et \; \left[N, N, N \right] \\
        \left[N, N, B \right] \; et \; \left[N, B, B \right] \\
        \left[B, B, B \right] \; et \; \left[R, R, B \right]
    \end{align*}
\textcolor{red}{\textbf{Stratégie}} \\
    On peut lister les multi-ensembles possibles en s'assurant d'écrire les 
    $R$ avant les $N$ et les $N$ avant $B$. etc ... \\
    En plaçant deux séparateurs entre les emplacements, on obtient : 
    \[ {5 \choose 2} \]
    Plus généralement, le nombre de multi-ensemble de $k$ éléments à partir d'un ensemble de $n$ 
    éléments est donné par : 
    \[ {{k+ n -1} \choose {n-1} } \]
    où $k$ représente la taille des multi-ensembles et $n$ le nombre de types d'objets. 

\end{EExample}

\begin{Definitionx*}{}{}
   Le nombre de \textbf{multi-ensembles} de  \textbf{k-éléments} qui peuvent être formé à partir 
   d'un ensemble de $n$ éléments $X = \{x_1, x_2, \dots, x_n \}$ est : 
   $ {k+n-1}\choose {k} $ $=$ ${k+n-1}\choose{n-1}$
   Cela fonction parce que n'importe quel multi-ensembles de \textbf{k-éléments} fait à partir de 
   $n$ éléments de $X$ est encodé en une liste d'\textit{étoiles} et de \textit{barres} de taille 
   $k+n -1$ ayant la forme : 


   \begin{table}[H]
     \begin{center}
       \renewcommand{\arraystretch}{1.5}
       \fontfamily{flr}\selectfont
       \footnotesize
           \begin{tabular}{l l l l l}
               \textit{ chaque $x_1$} & \textit{chaque $x_2$} 
                              & \textit{chaque $x_1$}
                              & & \textit{chaque $x_n$}
            \\
               $\overbrace{*\;\;*\;\;*\;\;*}|$ & $\overbrace{*\;\;*\;\;*\;\;*}|$ 
                                               & $\overbrace{*\;\;*\;\;*\;\;*}|$ 
                                               & $\cdots \; \cdots  |$ 
                                               & $\overbrace{*\;\;*\;\;*\;\;*}$
           \\
           \end{tabular}
   \end{center}
   \end{table}
   La liste encodé a la taille $k+n -1$ puisqu'il y a $k$ étoiles et $n-1$ barres qui séparent les 
   $n$ groupements d'étoiles. Une tele liste peut être obtenue en sélectionnant $n-1$ \textbf{positions}
   \textcolor{red}{\textit{parmi}} $n+k-1$  pour les barres, en remplissant les positions restantes 
   avec les étoiles. \textbf{Alternativement}, on peut sélectionner $k$ positions d'étoiles
   \textcolor{red}{\textit{parmi}} $k+n-1$ et remplir les positions restantes avec les barres. 

   les barres 
\end{Definitionx*}

\begin{Concept}{Pigeonnier}{}
Si $n$ objets sont placés dans $k$ cases tel que $n > k$, il y a forcément au moins une case 
qui contient plus d'un objet. 
\end{Concept}

\section{Principe de division (\S Hammack 4.9)}
        \paragraph{}
        Si $n$ \textbf{pigeons} vivent dans $k$ \textbf{boîtes} $(n \neq k)$, $k$ boîtes pourraient             
        être \textcolor{red}{\textit{vides}} et $k$ \textbf{boîtes} pourraient contenir plus d'un pigeon. 
        Le nombre moyen de pigeons par boîte est $\frac{n}{k}$. 
        
        \paragraph{}
        Par la définition d'une moyenne, 
        au moins une boîte doit contenir plus de $\frac{n}{k}$ pigeons et au moins une boîte doit contenir 
        moins de $\frac{n}{k}$ pigeons. Et parce qu'une boîte doit contenir un 
        \textcolor{red}{\textit{nombre entier}} de pigeons, on arrondit en disant qu'au moins une boîte 
        contient  $\lceil \frac{n}{k} \rceil$ pigeons ou plus. Par ailleurs, au moins  une boîte 
        contient  $\lfloor \frac{n}{k} \rfloor$ pigeons ou moins. 

\begin{Definitionx*}{Principe de division}{}
    Supposons que nous plaçons $n$ objets dans $k$ boîtes. Alors, au moins une boîte
    contient $\lceil \frac{n}{k} \rceil$ objets ou plus. Et au moins une boîte contient  
    $\lfloor \frac{n}{k} \rfloor$ pigeons ou moins
\end{Definitionx*}

\begin{Definitionx*}{Principe du Pigeonnier}{}
    Supposons que $n$ objets sont placés dans $k$ boîtes, \textbf{alors}, si $n > k$, il y a au moins une 
    boîte qui contient plus d'un objet. Si $n < k$ il y a au moins une boîte qui est vide. 
\end{Definitionx*}
\begin{EExample}{Classe}{}
    S'il y a $n \geq 60$ personnes dans l'auditoire, et que $k \leq 30$ jours dans un mois,
    par le principe du pigeonnier, on peut dire qu<il y a au moins deux personnes qui ont 
    le même jour de naissance. 
\end{EExample}

\begin{EExample}{pigeonnier et compression sans perte}{}
    Un fichier est simplement une liste de bits finie. Par ailleurs, un algorithme de compression 
    associe à chaque fichier un fichier unique. Le fichier résultant doit être unique.  \\\\
    Tout algoritme de compression compresse au moins un des fichiers en un fichier au moins aussi long.
    \\\\
    Soit $F_k$ l'ensemble des fichiers de $k$ bits, on a 
    \[ |F_k| = 2^k\]
    Soit $F_{<k}$ l<ensemble des fichiers de longueur $< k$, on a :
    \begin{align*}
             |F_{<k}| = |F_0 \cup F_1 \cup F_2 \cdots F_{k-1}| =       
    \end{align*}
    Si l'algorithme compressait tous les fhichier en des fichiers plus petits, il associeraient 
    à chacun des $2^k$ fichier de $F_k$ l'un des $2^{k-1}$ fichiers de $F_{<k}$. \\\\
    Par le principe du pigeonnier, au moins deux fichiers seraient compressés en le même fichier. Donc,
    un tel algorithme ne peut pas exister. 
\end{EExample}

\chapter{Probabilités discrètes} 

            

        \section{Probabilité d'évènement (\S Hammack 5.1)}
        \begin{Definitionx}{Expérience}{}
            Une \textbf{expérience} est une activité qui produit un des différents résultats possibles qui 
            ne peut pas être déterminé à l'avance. 
        \end{Definitionx}
        \begin{Definitionx*}{Univers de probabilité}{}
            L'ensemble des résultats possibles à une expérience est appelé \textcolor{red}{\textit{l'univers}}
            , noté $U$. Pour une expérience de lancé de dées 6 face, on a :
                                \[ U = \{1, 2, 3, 4, 5, 6 \} \] 
        \end{Definitionx*}
        \begin{Definitionx}{Evènement}{}
                Un évènement E est un sous-ensemble $E \subseteq U$ de résultats possibles. Par exemple, 
                $p = \{ 2, 4, 6 \}$. L'évènement se porduit si le résultat de \textbf{l'expérience} 
                apartient à $E$
        \end{Definitionx}

        \begin{Definitionx}{Probabilité}{}
            La \textbf{probabilité d'un évènement} $E$ est notée $P(E)$. Il s'agit de chances que l'évènement 
            se produise lorsqu'une expériuence est effectuée. 
            Si \textbf{tous} les résultats ont les 
            mêmes chances de se produire, \textbf{alors}, 
            \[ p(E) = \dfrac{|E|}{|U|} \]
        \end{Definitionx}

        \begin{EExample}{}{}
                On lance maintenant deux dés à $4$ faces l'un après l'autre. Quelle est la probabilité d'obtenir au 
                moins \textbf{un}   $2$ ?
                \[ U = Y \times Y  \]
              \begin{center}
              $\Big\{(1, 1), (1, 2), (1, 3), (1, 4),$ \\ 
              $(2, 1), (2, 2), (2, 3), (2, 4),$ \\ 
              $(3, 1), (3, 2), (3, 3), (3, 4),$\\ 
              $(4, 1), (4, 2), (4, 3), (4, 4) \Bigr\}$
              \end{center}
            Soit $D_2$ le fait d'obtenir a moins 2, on  a 
            $D_2 = \{ (a,b) \in U\;\; | \;\; a = 2 \text{ ou } b = 2 \}$  
            \[ P(D_2) = \dfrac{|D_2|}{|U|} = \dfrac{7}{16} \]
        \end{EExample}



        \section{Unions d'évènements (\S Hamnack 5.2)}

        \begin{Definitionx}{Occurence}{}
            En général, si $A$ et $B$ sont des évènement du même univers, \textbf{alors} 
            \textcolor{red}{$A \cup B$} est l'évènement où $A$ \textbf{ou} $B$ se produisent tout deux. 
            \textcolor{red}{$A \cap B$} est l'évènement où $A$ \textbf{et}  $B$ se produisent 
            simultanément. \textcolor{red}{$\overline{A}$} est l'évènement où $A$ \textbf{ne se produit pas}.     
        \end{Definitionx}

                \begin{note}{}{}
                    Lorsque l'intersection de deux évènements engendre un ensemble vide, $A \cap B = \emptyset$, 
                    cela veut dire que les évènements sont incompatibles. On peut aussi dire qu'ils sont 
                    \textcolor{red}{\textit{mutuellement exclusifs}}.   
                    \[ P(A \cup B) = P(A) + P(B) \]
                \end{note}


        \noindent Si on a deux évènements $A$ et $B$, l'évènement $A \cup B$ se produit si $A$ ou  ou $B$ se
        produit. En applicant le principe \textbf{d'inclusion exclusion}, on a : 
                \begin{align*}
                 P(A \cup B) = \dfrac{|A \cup B|}{|U|} \\ 
                = \dfrac{|A|}{|U|}+ \dfrac{|B|}{|U|} - \dfrac{|A - B|}{|U|} \\ 
                = P(A) + P(B) - P(A \cap B)
                \end{align*}

        
        \begin{Definitionx}{Évènement de non occurence}{}
            Similairement, on peut déterminer la probabilité d'un évènement de non occurence :
            \begin{align*}
                p(\overline{A}) = \dfrac{|\overline{A}|}{|U|} = \dfrac{|U - A|}{U} 
                                = \dfrac{|U|}{|U|} - \dfrac{|A|}{|U|}  
                                \\ =  1 - \dfrac{|A|}{|U|} = \textcolor{red}{1 - p(A)}
            \end{align*}
        \end{Definitionx}



        
        \section{Probabilité conditionnelle (\S Hammack 5.3)}

        
        \begin{Definitionx}{Probabilité conditionnelle}{}
            Si $A$ et $B$ sont deux évènements dans le même univers, la 
            \textbf{probabilité conditionnelle de A sachan B}, écrit $P(A|B)$, est la probabilité 
            que $A$ survienne si $B$ est déjà survenu. 
        \end{Definitionx}


    \begin{Definitionx}{Indépendance}{}
        Deux évènement sont \textbf{indépendant} si l'occurence de l'un ne change pas la probabilité 
        d'ocurrence de l'autre. Autrement dit :
        \[ p(A) = p(A|B), p(B) = p(B|A) \]
        Autrement, les deux évènements sont \textbf{dépendants}.   
    \end{Definitionx}


    \begin{Concept}{Probabilité conditionnelle}{}
        Supposons que $A$ ET $b$ sont des évènements dans le même univers. \textbf{Alors} : 
        \begin{enumerate}
            \item $p(A|B) = \dfrac{p(A \cup B)}{p(B)}$
            \item $p(B|A) = \dfrac{p(A \cup B)}{p(A)}$
            \item $p(A \cap B) = p(A|B)\cdot p(B) = p(A)\cdot p(B|A)$ 
            \item $p(A \cap B) = p(A)\cdot p(B)$ \dotfill si A et B sont \textbf{indépendant}.   
        \end{enumerate}
    \end{Concept}

    \section{Bayes (\S Hammack 5.5)}
    Supposons qu'un univers $U$ est divisé en deux sous univers tel que $U = U_1 \cup U_2$
    et $U_1 \cap U_2 = 0$, et il existe un évènement $E \subseteq U$. 
    La loi de Baye's permert de déterminer la probabilité que $U_1$ survienne si $E$ est survenu 
    \[ p(U_1|E)= \dfrac{p(U_1)\cdot p(E|U_1)}{p(U_1)\cdot p(E|U_1)+p(U_2)\cdot p(E|U_2)}\]

    \[ p(U_2|E)= \dfrac{p(U_1)\cdot p(E|U_2)}{p(U_1)\cdot p(E|U_1)+p(U_2)\cdot p(E|U_2)}\]

\end{multicols*}
\end{document}
